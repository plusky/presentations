\documentclass{beamer}

\batchmode

\usepackage{amsmath,amssymb,enumerate,epsfig,bbm,calc,color,ifthen,capt-of}

\usetheme{Berlin}
\usecolortheme{mit}

%% <language_settings> %%
\usepackage[utf8x]{inputenc}
\usepackage[czech]{babel}
%% </language_settings> %%

%% Source code insertions
\usepackage{verbatim}

%% <change me> %%
\title{Why use Gentoo}
\author[Tomáš Chvátal]{Tomáš Chvátal $<$scarabeus@gentoo.org$>$}
\date{2014/10/04}
%% </change me> %%

%% <logo> %%
\usepackage[absolute,overlay]{textpos}
\setlength{\TPHorizModule}{1mm}
\setlength{\TPVertModule}{1mm}
\newcommand{\MyLogo}{
	\begin{textblock}{14}(110,7)
		\includegraphics[width=14mm,height=14mm]{gentoo-logo.png}
	\end{textblock}
}

\newcommand*\oldmacro{}
\let\oldmacro\insertshorttitle
\renewcommand*\insertshorttitle{
	\MyLogo
	\oldmacro\hfill
}
%% </logo> %%

%% <bgimage> %%
\usebackgroundtemplate{
	\includegraphics[width=\paperwidth,height=\paperheight]{gentoo-background-1.png}
}
%% </bgimage> %%

%% <ToC Before each section> %%
%\AtBeginSection[]{
%	\begin{frame}<beamer>
%		\frametitle{Přehled}
%		\tableofcontents[currentsection]
%	\end{frame}
%}
%% </ToC Before each section> %%

%% <each bullet on one slide> %%
%% better to use \pause on points where we really want to stop :)
%\beamerdefaultoverlayspecification{<+->}
%% </each bullet on one slide> %%
% -----------------------------------------------------------------------------
\begin{document}
% -----------------------------------------------------------------------------
\frame{\titlepage}
%\section[Přehled]{}
%\begin{frame}{Přehled}
%	\tableofcontents
%\end{frame}
% -----------------------------------------------------------------------------
\section{Introduction}
\begin{frame}{Questions}
\begin{itemize}
	\item What is this Gentoo thing?
	\item I hate paying so much for central heating... Solutions?
	\item Optimalization overkill is always needed
\end{itemize}
\end{frame}

\begin{frame}{Who am I?}
	\begin{center}Tomáš Chvátal\end{center}
	\begin{itemize}
		\item Gentoo developer since fall 2008
		\item Was member of Gentoo Council for few terms
		\item Currently maintaining libreoffice (hint we need volunteers) and KDE team member
		\item Works at SUSE and is bossing around bunch of packagers :)
	\end{itemize}
\end{frame}
% -----------------------------------------------------------------------------
\section{What is Gentoo?}
\begin{frame}{Basic stuff}
	\begin{itemize}
		\item Gentoo was started by Daniel Robbins in 1999
		\item Metadistribution compiled from scratch on user system
		\item Supports loads of various archs: amd64, arm, x86, ppc64, ...
		\item Democratically governed distribution where Council stands as appealing body
		\item Portage contains something around 20 000 packages with 30 thousand versions
	\end{itemize}
\end{frame}

\begin{frame}{What the hell is Metadistribution?}
	Basically it means that Gentoo is used as base to create other distributions, instead of direct Gentoo usage.
	Some distros starting from Gentoo:
	\begin{itemize}
		\item Chrome OS
		\item Funtoo
		\item Calculate linux
		\item Sabayon
	\end{itemize}
	We can say Debian/Ubuntu are used like this (see all the damn forks with different skins there).
\end{frame}

\section{What is the cool stuff?}
\begin{frame}{Rolling distribution}
	\begin{itemize}
		\item Always up-to-date
		\item No need for migration
		\item New stuff every 30 minutes
	\end{itemize}
\end{frame}

\begin{frame}{Build locally from source code}
	\begin{itemize}
		\item Optimize directly for your hardware
		\item Tweak out dependencies and configure options for build
	\end{itemize}
\end{frame}

\begin{frame}{Hardened project}
	\begin{itemize}
		\item Super secure Linux TM
		\item Enables quite some compilation switches to harden the output
		\item Various kernel patches to skyrocket the security
	\end{itemize}
\end{frame}

\begin{frame}{Other trivia}
	\begin{itemize}
		\item Ability to tweak out options a lot (no cups, no systemd, etc.)
		\item Almost no distro patches: you get the package as upstream desires
		\item You can choose to live on the edge by using testing tree
	\end{itemize}
\end{frame}

\section{Not so cool stuff}

\begin{frame}{What should I be aware of}
	\begin{itemize}
		\item Quite complex installation compared to others
		\item Rolling updates
		\item Slow application install
		\item Testing tree can be fatal and you might be forced to reinstall
	\end{itemize}
\end{frame}

\section{When  should I use it}
\begin{frame}{When should I start considering to install it?}
	\begin{itemize}
		\item I want to learn really well how this linux thing works
		\item I don't mind searching answers on internet
		\item I actually read distro notification and act upon them
	\end{itemize}
	Mostly Gentoo can be deployed everywhere, just some administration decisions can make it really big nightmare.
\end{frame}

\section{Contributing}
\begin{frame}{I feel so intrigued I want to contribute}
	\begin{itemize}
		\item Report bugs to http://bugs.gentoo.org/
		\item Contribute to portage (became developer or contribute like proxy-maintainer)
		\item Finances (donate button on http://www.gentoo.org/)
	\end{itemize}
\end{frame}

% -----------------------------------------------------------------------------
\section{Endnote}

\subsection{SUSE is Hiring!}

\begin{frame}{We are hiring at SUSE}
	\begin{center}\bf{We are looking for people to have fun at SUSE!}\end{center}
	\begin{center}\bf{Drop by SUSE booth to get informations.}\end{center}
\end{frame}

\subsection{More reading}
\begin{frame}{More reading}
	\begin{itemize}
		\item http://www.gentoo.org/
		\item http://forums.gentoo.org/
		\item http://wiki.gentoo.org/
	\end{itemize}
\end{frame}

\subsection{Questions}
\begin{frame}{Q\&A}
	\begin{center}Do you happen to have some questions people?\end{center}
\end{frame}

\subsection{Thanks}
\begin{frame}{}
	\begin{center}Thank you for your attention\end{center}
\end{frame}
% -----------------------------------------------------------------------------
\end{document}
