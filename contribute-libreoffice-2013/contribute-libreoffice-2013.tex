\documentclass{beamer}
\usetheme[pageofpages=of,% String used between the current page and the
                         % total page count.
          bullet=circle,% Use circles instead of squares for bullets.
          titleline=true,% Show a line below the frame title.
	  titlepagelogo=opensuse,
          alternativetitlepage=true,% Use the fancy title page.
          ]{Torino}

\setbeamerfont{title}{series=\bfseries,size=\LARGE}
\author{Tom\'{a}\v{s} Chv\'{a}tal\newline {\small SUSE QA Maintenance}}
\title{Contributing to LibreOffice}

\begin{document}

\begin{frame}[t,plain]
\titlepage
\end{frame}

\section{Why contribute}

\begin{frame}{Why contribute}
	\begin{itemize}
	\item Huge interesting codebase.
	\item Easy tasks for begginers to ease up the start.
	\item Chance to score a cool job.
	\item Even small fixes are visible fast.
	\end{itemize}
\end{frame}

\section{How contribute}

\begin{frame}{Contribute - bugzilla}
	\begin{itemize}
	\item Testing old reports if they are still valid.
	\item Testing regressions to find the culprit with help of bibisect. https://wiki.documentfoundation.org/QA/HowToBibisect
	\item Finding duplicates.
	\end{itemize}
\end{frame}

\begin{frame}{Contribute - code}
	When you are contributing to LibreOffice you should clarify under which license you are providing your patches. Preffered
	way of telling that is sending mail to libreoffice-dev@freedesktop.org containing something like "All of my past \& future
	contributions to LibreOffice may be licensed under the MPL/LGPLv3+ dual license." \\
	After that you can pick two ways of submitting patches:
	\begin{itemize}
	\item Providing the patch to libreoffice-dev mailinglist in git format (git format-patch/git send/...).
	\item Use gerrit instance and recieve feedback there. https://wiki.documentfoundation.org/Development/gerrit
	\end{itemize}
\end{frame}
\begin{frame}{Contribute - code pt2}
	With your patch in the mailinglist/gerrit someone from the core team will do the review. After you fix all the issues
	he might find the patch gets added to master branch. \\
	There is also approach for cherry-picks where you just open the gerrit request with the selected commit range against
	desired branch. When it obtains enough reviews it gets merged.
\end{frame}

\begin{frame}{Contribute - testing}
	\begin{itemize}
	\item Running tinderbox instance. https://wiki.documentfoundation.org/Development/Tinderbox
	\item Writting automated tests or manual testing of betas.
	\item Localization QA.
	\end{itemize}
\end{frame}

\begin{frame}{Contribute - \$\$\$}
	\begin{itemize}
	\item Of course The Document Foundation will be more than gratefull if you decide to donate any amount to support the development. \\
	\item Other option is to donate cpu time running the tinderbox instances for gerrit buildbot testing. There is no need to give the machine away.
	You just have to run the instance and register it in the wiki above.
	\end{itemize}
\end{frame}

\subsection{Endnote}
\begin{frame}{}
        \begin{center}
		Thanks for the attention.\\
		Feel free to ask almost anything...
	\end{center}
\end{frame}

\end{document}

