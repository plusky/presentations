\documentclass{beamer}
\usepackage[utf8x]{inputenc}
\usepackage[czech]{babel}
\usetheme[pageofpages=of,% String used between the current page and the
                         % total page count.
          bullet=circle,% Use circles instead of squares for bullets.
          titleline=true,% Show a line below the frame title.
	  titlepagelogo=opensuse,
          alternativetitlepage=true,% Use the fancy title page.
          ]{Torino}

\setbeamerfont{title}{series=\bfseries,size=\LARGE}
\author{Tom\'{a}\v{s} Chv\'{a}tal\newline {\small openSUSE Team}}
\title{Autotools workshop}
\date{2013/07/19}

\begin{document}

\begin{frame}[t,plain]
\titlepage
\end{frame}

\section{Introduction}

\begin{frame}[t]{Who the hell is Tomáš Chvátal}
	\begin{itemize}
	\item SUSE Employee since 2011 (QA, openSUSE)
	\item Packager of Libreoffice and various other stuff for openSUSE
	\item openSUSE promoter and volunteer
	\item Gentoo developer since fall 2008 and Council member since 2010
	\end{itemize}
	\begin{center}As this is workshop, remember to ask questions if in doubt or curious. I don't bite, really.\end{center}
\end{frame}

\section{Autotools description}

What does automake/autoconf actually do and generate

\section{Autoconf layout}

Setting version

Finding compiler

Pkgconfig finding and setting ; why is it bad to ; 0.28 can crosscompile;

Options/automagicness

\section{m4 macros}
(boost, easy mutliple same switches [libreoffice])

\section{Automake layout}

Why rootdir makefile rule them all

Adding libraries

Adding sources

Avoiding globs

Using make distcheck to verify the pre-release quality

Release hooks

\section{Libtool}

\section{Writing simple build system}
https://github.com/openSUSE/sax3/tree/kill-ui-library

\section{More complex samples and discussion}
libvisio
libtextextcat

\section{Endnote}

\section{Thanks}

\begin{frame}{Suse is hiring}
	\begin{figure}
	\includegraphics[width= 0.8\linewidth]{suse_hiring.png}
	\end{figure}
\end{frame}

\begin{frame}{Thanks}
	\begin{center}
	Thank you for your attention.
	\end{center}
\end{frame}

\end{document}

