\documentclass{beamer}

\batchmode

\usepackage{amsmath,amssymb,enumerate,epsfig,bbm,calc,color,ifthen,capt-of}

\usetheme{Berlin}
\usecolortheme{mit}

%% <language_settings> %%
\usepackage[utf8x]{inputenc}
\usepackage[czech]{babel}
%% </language_settings> %%

%% Source code insertions
\usepackage{verbatim}

%% <change me> %%
\title{Vlastnosti hardened kernelu pro každého paranoika}
\author[Tomáš Chvátal]{Tomáš Chvátal $<$scarabeus@gentoo.org$>$}
\date{2013/06/01}
%% </change me> %%

%% <logo> %%
\usepackage[absolute,overlay]{textpos}
\setlength{\TPHorizModule}{1mm}
\setlength{\TPVertModule}{1mm}
\newcommand{\MyLogo}{
	\begin{textblock}{14}(110,7)
		\includegraphics[width=14mm,height=14mm]{gentoo-logo.png}
	\end{textblock}
}

\newcommand*\oldmacro{}
\let\oldmacro\insertshorttitle
\renewcommand*\insertshorttitle{
	\MyLogo
	\oldmacro\hfill
}
%% </logo> %%

%% <bgimage> %%
\usebackgroundtemplate{
	\includegraphics[width=\paperwidth,height=\paperheight]{gentoo-background-1.png}
}
%% </bgimage> %%

%% <ToC Before each section> %%
%\AtBeginSection[]{
%	\begin{frame}<beamer>
%		\frametitle{Přehled}
%		\tableofcontents[currentsection]
%	\end{frame}
%}
%% </ToC Before each section> %%

%% <each bullet on one slide> %%
%% better to use \pause on points where we really want to stop :)
%\beamerdefaultoverlayspecification{<+->}
%% </each bullet on one slide> %%
% -----------------------------------------------------------------------------
\begin{document}
% -----------------------------------------------------------------------------
\frame{\titlepage}
%\section[Přehled]{}
%\begin{frame}{Přehled}
%	\tableofcontents
%\end{frame}
% -----------------------------------------------------------------------------
\section{Představení}

\subsection{Kdo je Tomáš Chvátal}

\begin{frame}{Kdo je Tomáš Chvátal}
	\begin{itemize}
		\item Vývojář Gentoo od podzima 2008
		\item Člen councilu od ledna 2010
		\item Člen KDE týmu (chvíli jim i šéfoval než se objevilo akonadi)
		\item Správce/vývojář LibreOffice
		\item Dříve také v Gentoo pracoval na X11, Overlays, Clustering, QA, ...
		\item Pracuje v SUSE jako L3/QA Maintenance
	\end{itemize}
	\begin{center}To byl nápad tu prezentaci dělat česky. Složitější odborné termíny budou v angličtině jinak bysme se z toho zbláznili.\end{center}
\end{frame}
% -----------------------------------------------------------------------------
\section{O co se jedná v Hardened Gentoo}

\subsection{Základ Hardened Gentoo}

\begin{frame}{Základní informace}
	\begin{itemize}
		\item Projekt pro zvýšení zabezpečení počítače pomocí různých patchů (viz další slide)
		\item Snahou je co nejvíce těchto vlastností integrovat přímo do hlavniho profilu Gentoo
		\item Z důvodu snížení výkonu některých aplikací a zamezení fuknčnosti některých funkcí pro desktop je to stále oddělený projekt
	\end{itemize}
	\begin{center}http://www.gentoo.org/proj/en/hardened/\end{center}
\end{frame}

\begin{frame}{Dostupné funkce}
	\begin{itemize}
		\item Nastavení toolchainu (kompiler, linker, .. ) jako vynucení PIE, kontrola zásobníků při kompilaci, nebo ochrana proti stack-smashingu
		\item Rozšíření jádra PaX, poskytující non-executable memory, address space layout randomization, \ldots
		\item Rozšíření jádra grSecurity, umožňující restrikce chrootu, dodatečný audit, omezení procesů, \ldots
		\item Rozšíření jádra SELinux, MAC (Mandatory Access Control) rozšiřující běžná omezení linuxových práv
		\item Technologie komem Integrity, jako Integrity Measurement Architecture, která chrání systém proti nevítaným změnám
	\end{itemize}
\end{frame}

\section{Lehčí detaily jednotivých funkcí}

\subsection{Toolchain}

\begin{frame}{Zabezpečení při kompilaci}
	\begin{center}PIE a FORTIFY\_SOURCE jsou i v základním profilu\end{center}
	\begin{itemize}
		\item -DFORTIFY\_SOURCE=2: zabezpečení proti jednoduchým přetečením zásobníků
		\item PIE/PIC: kód nezávislý na pozici v paměti, vetšina skoků je počítána tedy relativne místo abs. odkazů
		\item SSP: ochrana proti stack-smashing z GCC, přidá na konec (lze i náhodně) zásobníku kanárka který informuje o pokusu o přetečení ať náhodném či cíleném (sranda sledovat s nepomukem)
	\end{itemize}
\end{frame}

\begin{frame}{Zabezpečení při linkování}
	\begin{itemize}
		\item -Wl,-z,relro: označí části knihovny pouze pro čtení a znemožní úpravy (GOT, PLT)
		\item -Wl,-z,now: přeloží všechny symboly a vynutí načtení knihoven při spuštení aplikace (spadne když nejsou všechny splněny) a neznatelně zpomalí první spuštění applikace
	\end{itemize}
\end{frame}

\subsection{grSecurity}

\begin{frame}{grSecurity}
	\begin{itemize}
		\item RBAC část: rozšíření unixových přístupových práv o další možnosti, např. ochrana před brute-force, skrytí ptrace vybraným procesům, ...
		\item Omezení chroot: ochrana proti priv-esc a další omezení/zábrany: zamezený přístup do sdílené paměti z chrootu, nemožnost videt procesy mimo chroot, omezený kill/sgid/...
		\item Audit: logovaní činností uživatelů, mount, změny času, použití chdir, zaznamenání příkazů Exec, nezdařené fork...
	\end{itemize}
\end{frame}

\begin{frame}{grSecurity - nastavení}
	grSecurity obsahuje spoustu možností a vyplatí se je nastudovat s webových stránek projektu. \\
	Gentoo se snaží proti výchozím možnostem (Nízké/Vysoké zabezpečení) přidat ještě možnost desktop/server, kdy jsou ty nejzajímavější možnosti povoleny.\\
	Mimo jádro už se moc věcí pro grSec dělat nemusí, spíše se jedná o nastavení pro PaX.
\end{frame}

\subsection{PaX}

\begin{frame}{PaX}
	Technicky vzato se jedná o část grSecurity, která není vyvýjená upstreamem a umožnuje následující činnosti.
	\begin{itemize}
		\item ASLR: náhodné rozmístění adresového prostoru a proto útočník neodhadne rozvržení paměti
		\item Vynucení stavu paměti: buď je ke čtení nebo pouze k zápisu. VELICE zpomalí systém, zato ho zatraceně dobře zabezpečí (binarní drivery pláčou)
		\item Trampolínky: runtime rozšíření pro SSP dá se říct, protože dělá téměř to samé. Bohužel runtime ovládá PaX a tedy díra v PaX kompromituje celý systém
	\end{itemize}
\end{frame}

\begin{frame}[fragile]{PaX - ovládání}
	Doporučuji na testování stáhnou soubor checksec.sh a nainstalovat paxtest.
	\begin{itemize}
		\item paxctl -flagy binarka / paxctl-ng -flagy binarka
		\item Paxctl zapisuje přímo do elf a nefunguje např. nefunguje na Skype
		\item Paxctl-ng používá xattr (v Gentoo myslím to používá i starý pax)
	\end{itemize}
	\begin{small}
	\begin{verbatim}
root@desktopik: ~ # paxctl-ng -v /usr/lib64/libreoffice/program/soffice.bin
/usr/lib64/libreoffice/program/soffice.bin:
        PT_PAX   : -em--
        XATTR_PAX: not found
	\end{verbatim}
	\end{small}
\end{frame}

\begin{frame}[fragile]{PaX - výstup checksec.sh}
	\begin{tiny}
	\begin{verbatim}
...
* Does the CPU support NX: Yes

         COMMAND    PID RELRO             STACK CANARY           NX/PaX        PIE
            init      1 Full RELRO        Canary found           NX enabled    PIE enabled
           udevd   2583 Full RELRO        Canary found           NX enabled    PIE enabled
     dbus-daemon   3176 Full RELRO        Canary found           NX enabled    PIE enabled
        rsyslogd   3190 Full RELRO        Canary found           NX enabled    PIE enabled
 console-kit-dae   3209 Full RELRO        Canary found           NX enabled    PIE enabled
         polkitd   3283 Full RELRO        Canary found           NX enabled    PIE enabled
  wpa_supplicant   3520 Full RELRO        Canary found           NX enabled    PIE enabled
         wpa_cli   3527 Full RELRO        Canary found           NX enabled    PIE enabled
          smartd   4089 Full RELRO        No canary found        NX enabled    PIE enabled
               X   3564 Partial RELRO     Canary found           NX enabled    PIE enabled
...
	\end{verbatim}
	\end{tiny}
\end{frame}

\begin{frame}[fragile]{PaX - výstup paxtest}
	\begin{tiny}
	\begin{verbatim}
# paxtest
Executable anonymous mapping             : Killed
Executable bss                           : Killed
Executable data                          : Killed
Executable heap                          : Killed
Executable stack                         : Killed
Executable anonymous mapping (mprotect)  : Killed
Executable bss (mprotect)                : Killed
Executable data (mprotect)               : Killed
Executable heap (mprotect)               : Killed
Executable stack (mprotect)              : Killed
Executable shared library bss (mprotect) : Killed
Executable shared library data (mprotect): Killed
Writable text segments                   : Killed
Anonymous mapping randomisation test     : 16 bits (guessed)
Heap randomisation test (ET_EXEC)        : 13 bits (guessed)
Heap randomisation test (ET_DYN)         : 25 bits (guessed)
Main executable randomisation (ET_EXEC)  : 16 bits (guessed)
Main executable randomisation (ET_DYN)   : 17 bits (guessed)
Shared library randomisation test        : 16 bits (guessed)
Stack randomisation test (SEGMEXEC)      : 23 bits (guessed)
Stack randomisation test (PAGEEXEC)      : No randomisation
Return to function (strcpy)              : Vulnerable
Return to function (memcpy)              : Vulnerable
Return to function (strcpy, RANDEXEC)    : Killed
Return to function (memcpy, RANDEXEC)    : Killed
Executable shared library bss            : Killed
Executable shared library data           : Killed
	\end{verbatim}
	\end{tiny}
\end{frame}


\subsection{SELinux}

\begin{frame}{SELinux}
	\begin{itemize}
		\item O SELinuxu nevím téměř níc a zvládl jsem to nastavit pouze jednou
		\item Doporučuji přečíst si dokumentaci a Svenův blog
	\end{itemize}
	\begin{center}http://www.gentoo.org/proj/en/hardened/selinux/selinux-handbook.xml\end{center}
	\begin{center}http://blog.siphos.be/category/gentoo/hardened/\end{center}
\end{frame}

\subsection{Integrity}

\begin{frame}{Advanced Intrusion Detection Environment}
	\begin{itemize}
		\item Jedná se o metodu detekce průniku (AIDE).
		\item V Gentoo je balíček dostupný jako app-forensics/aide.
		\item Je důležité si správně nastavit co vše sledovat (ani málo ani moc)
		\item Nastavení musí být read-only pokudmožno externě (nfs?)
		\item Skenování by se mělo provádět offline z livecd/memory-sticku
	\end{itemize}
\end{frame}

\begin{frame}[fragile]{AID - ukázkový výstup}
	\begin{tiny}
	\begin{verbatim}
AIDE found differences between database and filesystem!!
Start timestamp: 2013-05-30 16:41:02

Summary:
  Total number of files:        625
  Added files:                  0
  Removed files:                0
  Changed files:                2


---------------------------------------------------
Changed files:
---------------------------------------------------

changed: /etc/pam.d/
changed: /etc/pam.d/sudo
	\end{verbatim}
	\end{tiny}
\end{frame}


\begin{frame}[fragile]{AID - ukázkový výstup - page 2}
	\begin{tiny}
	\begin{verbatim}
---------------------------------------------------
Detailed information about changes:
---------------------------------------------------

Directory: /etc/pam.d
  Mtime    : 2013-05-11 21:09:20              , 2013-05-30 16:01:02
  Ctime    : 2013-05-11 21:09:20              , 2013-05-30 16:01:02

File: /etc/pam.d/sudo
  Size     : 135                              , 80
  Mtime    : 2013-05-11 21:09:20              , 2013-05-30 16:01:02
  Ctime    : 2013-05-11 21:09:20              , 2013-05-30 16:01:02
  Inode    : 328303                           , 464053
  MD5      : 239be3ac285c0860e5e81a==         , eLUrP2BKw43eExAZX+dlBA==
  SHA1     : e7d7393f0768ed2dbebdBne5V6E=     , KwQ42poukMiqEjKQ7e9xkBNZB8=
	\end{verbatim}
	\end{tiny}
\end{frame}

\section{Menší zamyšlení závěrem}

\subsection{Nejpravděpodobnější možnosti útoků}

\begin{frame}{Nejpravděpodobnější možnosti útoků na desktop v dnešní době}
	\begin{itemize}
		\item Zneužití špatně nastavených pravidel consolekit/dbus
		\item Zneužití špatně nastaveného d-bus systému
		\item SUID binárka liknovaná s kreativníma knihovnama (Xlib, ...), přecijen suid dává kernel
		\item pomocí podvržení balíčku (distro od distra podle toho jak mají řešené podpisy)
	\end{itemize}
\end{frame}

\subsection{Obrana}

\begin{frame}{Obrana}
	\begin{itemize}
		\item Přinucení distribucí vracet upstreamu patche s výchozím chováním, které je bezpečné
		\item Díky předchozímu zabezpečení vetšiny distribucí
		\item Odebírání suid bitů kde jen to jde
		\item Více paranoiků kteří pomáhají s položkou číslo 1
	\end{itemize}
\end{frame}

% -----------------------------------------------------------------------------
\section{Souhrn}

\subsection{Souhrn}

\begin{frame}{Dotazy}
	\begin{center}Otázky a odpovědi, koukejte se na něco zeptat, měli bysme na tuhle část mít 5 minut.\end{center}
\end{frame}

\begin{frame}{Poděkování}
	\begin{center}Děkuji za pozornost\end{center}
\end{frame}
% -----------------------------------------------------------------------------
\end{document}
