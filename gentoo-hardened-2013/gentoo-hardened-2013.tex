\documentclass{beamer}

\batchmode

\usepackage{amsmath,amssymb,enumerate,epsfig,bbm,calc,color,ifthen,capt-of}

\usetheme{Berlin}
\usecolortheme{mit}

%% <language_settings> %%
\usepackage[utf8x]{inputenc}
\usepackage[czech]{babel}
%% </language_settings> %%

%% Source code insertions
\usepackage{verbatim}

%% <change me> %%
\title{Installing or migrating to Hardened Gentoo}
\author[Tomáš Chvátal]{Tomáš Chvátal $<$scarabeus@gentoo.org$>$}
\date{2013/03/02}
%% </change me> %%

%% <logo> %%
\usepackage[absolute,overlay]{textpos}
\setlength{\TPHorizModule}{1mm}
\setlength{\TPVertModule}{1mm}
\newcommand{\MyLogo}{
	\begin{textblock}{14}(110,7)
		\includegraphics[width=14mm,height=14mm]{gentoo-logo.png}
	\end{textblock}
}

\newcommand*\oldmacro{}
\let\oldmacro\insertshorttitle
\renewcommand*\insertshorttitle{
	\MyLogo
	\oldmacro\hfill
}
%% </logo> %%

%% <bgimage> %%
\usebackgroundtemplate{
	\includegraphics[width=\paperwidth,height=\paperheight]{gentoo-background-1.png}
}
%% </bgimage> %%

%% <ToC Before each section> %%
%\AtBeginSection[]{
%	\begin{frame}<beamer>
%		\frametitle{Přehled}
%		\tableofcontents[currentsection]
%	\end{frame}
%}
%% </ToC Before each section> %%

%% <each bullet on one slide> %%
%% better to use \pause on points where we really want to stop :)
%\beamerdefaultoverlayspecification{<+->}
%% </each bullet on one slide> %%
% -----------------------------------------------------------------------------
\begin{document}
% -----------------------------------------------------------------------------
\frame{\titlepage}
%\section[Přehled]{}
%\begin{frame}{Přehled}
%	\tableofcontents
%\end{frame}
% -----------------------------------------------------------------------------
\section{Introduction}
\begin{frame}{Who the hell is Tomáš Chvátal}
	\begin{itemize}
		\item Gentoo developer since fall 2k8
		\item Council member since Jan 2010
		\item KDE team member (former lead)
		\item Libreoffice maintainer and also upstream developer
		\item Formely also working on X11, Overlays, Clustering, QA, ...
	\end{itemize}
\end{frame}
% -----------------------------------------------------------------------------
\section{What is gentoo hardened}

\begin{frame}{Basic informations}
	\begin{itemize}
		\item Gentoo project focusing on providing latest security measures.
		\item Most of the features are accepted to the main project
		\item The reason for separate project is that security can have speed inpact and make maintenance bit harder
	\end{itemize}
	\begin{center}More to be described later in the presentation.\end{center}
\end{frame}

\section{Install and migration}

\begin{frame}{Install from scratch}
	\begin{center}At first the plan was just to show up the instalation process but during the at home tryout i found out that it takes 5 hours from scratch.\end{center}
		The difference is only in one step of the normal guide http://www.gentoo.org/doc/en/handbook/index.xml \\
		You have to use stage3 at: http://distfiles.gentoo.org/releases/amd64/autobuilds/current-stage3/hardened/ \\
		And when prompted for setting the profile by eselect, you just use hardened one.
\end{frame}

\begin{frame}{Migration}
	\begin{center}I expected this to be bit complex, but the results are contrary my assesments, behold these are the only needed steps:\end{center}
	\begin{itemize}
		\item eselect profile list
		\item eselect profile set [number of hardened profile]
		\item emerge -1v binutils gcc virtual/libc
		\item gcc-config -l
		\item gcc-config [the gcc from hardened profile]
		\item emerge -1v gcc virtual/libc
		\item source /etc/profile
		\item emerge -1evD @world
	\end{itemize}
\end{frame}

\begin{frame}{Desktop troubles}
	There are two issues when migrating your desktop:
	\begin{itemize}
		\item Usage of binary drivers where nvidia does not work at all and fglrx only with crazy hacks.
		\item Some PAX options does not allow running X at all, broken features are: CONFIG\_PAX\_KERNEXEC, CONFIG\_GRKERNSEC\_IO.
	\end{itemize}
	\begin{itemize}
		\item CONFIG\_GRKERNSEC\_IO: ioperm and iopl calls return error when enabled, but X won't start at all.
		\item CONFIG\_PAX\_KERNEXEC: protects agains code injections into kernel memory, but causes input devices like mouse to hang.
	\end{itemize}
\end{frame}

\section{Hardened features}

\begin{frame}{Hardened features}
	We can split the features into 4 indepenednt areas:
	\begin{itemize}
		\item Compilation protection
		\item Link time protection
		\item Kernel level
		\item GCC plugins from kernel
	\end{itemize}
\end{frame}

\subsection{Compilation}

\begin{frame}{Stack-Smashing Protector}
	\begin{center}
	\begin{tabular}{|c|c|}
		\hline
		SSP & NORMAL GCC \\
		\hline
		Variable n & Variable n \\
		\ldots & \ldots \\
		Variable 1 & Variable 1 \\
		Argument n & Argument n \\
		\ldots & \ldots \\
		Argument 1 & Argument 1 \\
		Return addr & Return addr \\
		Canary & Local variables \\
		Local variables & --- \\
		\hline
	\end{tabular}
	\end{center}
\end{frame}

\begin{frame}{-DFORTIFY\_SOURCE=2}
	\begin{center}
		Various operations detecting software vulnerabilities in compile and run time.
	\end{center}
\end{frame}

\begin{frame}{PIE/PIC}
	\begin{center}
	Position-independent code/executable is code that can be put anywhere in the memory and does not require specific absolute address alocation.
	\end{center}
\end{frame}

\subsection{Linking}

\begin{frame}{RELRO}
	\begin{center}
	Linker marks some critical sections of the object like GOT and PLT tables as read only.
	\end{center}
	You can use this with linker flags -Wl,-z,relro \\
	PLT means Procedure Linkage Table. \\
	GOT stands for Global Offset Tables.
\end{frame}

\begin{frame}{BIND\_NOW}
	\begin{center}
	Ensures that all symbols ale loaded during aplication start so we can have GOT and PLT tables completely read-only. Sideefect is that apps can take more time to fire up.
	\end{center}
	You can use this with linker flags -Wl,-z,now
\end{frame}

\subsection{Kernel GCC plugins}
\begin{frame}{GCC plugins}
	\begin{itemize}
		\item kernexec: forces function pointer calls and function returns to jump into kernel space.
		\item constify: maks the ops structs and structs using do\_const parm as read only at compile time (which is implied by this for runtime too).
		\item stackleak: sanitizes kernel stack by providing estimate of how much deep was the kernel stack used by the syscall.
		\item interoverflow: detects calls which could have suffered from under/overflow by detecting these and prevents usage of incorrect sizes.
	\end{itemize}
\end{frame}

\subsection{Kernel features}

\begin{frame}{Features part 1}
	\begin{itemize}
	\item ASLR, RANDMMAP: Safeguards that provide random adresses to mmap calls unless required otherwise by the call.
	\item RANDSTACK: Inserts random adress for the stack between system calls.
	\item /proc only for user: Restricts the access to process informations only to owner (and group if needed).
	\item Brute force: If daemon aobrts or gets interesting signal it can't fork during next 30 seconds.
	\item UDEREF: prevents accessx to user space functions and data.
	\item MIN\_MMAP\_ADDR: forbids memory reservation on lower adresses.
	\end{itemize}
\end{frame}

\begin{frame}{Features part 2}
	\begin{itemize}
	\item Memory sanitization: wipes mem pages when they are freed so nobody can read stored data.
	\item Kernel stack sanitization: wipes the stack before returning from syscall.
	\item Dmesg restriction: nobody except root can run dmesg.
	\item Kernel symbol hiding: the symbols table is readable only by root and SYS\_MODULE can provide the needed info via syscalls.
	\item Kernel process hiding: prevents user from seeing kernel threads.
	\end{itemize}
\end{frame}

\begin{frame}{TODO}
NX memory
Mprotect restriction
Reference counter overflow protection
Free mem sanitization
Kernel Stack sanitization
VM86 mode restriction
Disabled privileged I/O
sysfs/debugfs restrictions
linking restrictions
fifo restrictions
chroot jail restrictions
enforce process limit on execs (not only on forks)
dmesg restrictions
TCP/UDP blackholing and LAST\_ACK DoS protection
Code execution on non trusted folders
ptrace denial on non readable sugid binaries
socket restrictions
consistent multitrheaded privilege enforcement
active kernel exploit responce
kernel auditing
\end{frame}

% -----------------------------------------------------------------------------
\section{Wrap-up}

\subsection{Wrap-up}

\begin{frame}{Links to insteresting stuff about hardened}
	\begin{itemize}
		\item Project page - http://www.gentoo.org/proj/en/hardened/
		\item Svens blog - http://blog.siphos.be/category/gentoo/hardened/
	\end{itemize}
\end{frame}

\begin{frame}{Questions}
	\begin{center}Nobody wants to ask anything. Right?\end{center}
\end{frame}

\begin{frame}{Thank you}
	\begin{center}Thank you for your attention!\end{center}
\end{frame}
% -----------------------------------------------------------------------------
\end{document}
