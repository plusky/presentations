\documentclass{beamer}

\batchmode

\usepackage{amsmath,amssymb,enumerate,epsfig,bbm,calc,color,ifthen,capt-of}

\usetheme{Berlin}
\usecolortheme{mit}

%% <language_settings> %%
\usepackage[utf8x]{inputenc}
\usepackage[czech]{babel}
%% </language_settings> %%

%% Source code insertions
\usepackage{verbatim}

%% <change me> %%
\title{Contributing to Gentoo: Getting Involved more}
\author[Markos Chandras \& Tomáš Chvátal]{Markos Chandras $<$hwoarang@gentoo.org$>$ \\ Tomáš Chvátal $<$scarabeus@gentoo.org$>$}
\date{2012/10/21}
%% </change me> %%

%% <logo> %%
\usepackage[absolute,overlay]{textpos}
\setlength{\TPHorizModule}{1mm}
\setlength{\TPVertModule}{1mm}
\newcommand{\MyLogo}{
	\begin{textblock}{14}(110,7)
		\includegraphics[width=14mm,height=14mm]{gentoo-logo.png}
	\end{textblock}
}

\newcommand*\oldmacro{}
\let\oldmacro\insertshorttitle
\renewcommand*\insertshorttitle{
	\MyLogo
	\oldmacro\hfill
}
%% </logo> %%

%% <bgimage> %%
\usebackgroundtemplate{
	\includegraphics[width=\paperwidth,height=\paperheight]{gentoo-background-1.png}
}
%% </bgimage> %%

%% <ToC Before each section> %%
%\AtBeginSection[]{
%	\begin{frame}<beamer>
%		\frametitle{Přehled}
%		\tableofcontents[currentsection]
%	\end{frame}
%}
%% </ToC Before each section> %%

%% <each bullet on one slide> %%
%% better to use \pause on points where we really want to stop :)
%\beamerdefaultoverlayspecification{<+->}
%% </each bullet on one slide> %%
% -----------------------------------------------------------------------------
\begin{document}
% -----------------------------------------------------------------------------
\frame{\titlepage}
%\section[Přehled]{}
%\begin{frame}{Přehled}
%	\tableofcontents
%\end{frame}
% -----------------------------------------------------------------------------
\section{Představení}
\begin{frame}{Kdo jsem?}
	\begin{center}Tomáš Chvátal\end{center}
	\begin{itemize}
		\item Vývojářem Gentoo od podzima 2008
		\item Členem Gentoo Council od ledna 2010
		\item V současnosti správce libreoffice a člen KDE týmu
		\item Dříve členem QA, X11, Overlays, Clustering, ...
	\end{itemize}
\end{frame}
% -----------------------------------------------------------------------------
\section{Co je gentoo?}

\begin{frame}{Základní údaje}
	\begin{itemize}
		\item Zahájena Danielem Robbinsem v roce 1999.
		\item Metadistribuce skládaná ze zdrojových kódů
		\item Podporuje mnoho architektur: amd64, arm, x86, ppc64, ...
		\item Demokraticky řízená distribuce s Koncilem jako rozhodčím orgánem
		\item V portage je neco kolem 20 000 balíčků se 30 tisíci verzemi, další bordel je v overlays
	\end{itemize}
\end{frame}

\begin{frame}{Co to sakra znamená Metadistribuce?}
	V podstatě se jedná o to, že většina distribucí vychází z Gentoo jako základu a uživatelé používají tyto verze místo přímého používání Gentoo.
	Distribuce, které vycházejí z Gentoo:
	\begin{itemize}
		\item Sabayon
		\item Funtoo
		\item Calculate linux
		\item Exherbo (nepoužívá portage pouze použilo Gentoo jako inspiraci)
	\end{itemize}
	Další podobnou distribucí je Debian/Ubuntu, kdy si každej \uv{blbec} bastlí svojí vlastní verzi (viz Greenie linux).
\end{frame}

\begin{frame}{Výhody proti binárním distribucím}
	\begin{itemize}
		\item Rolling-updates: distribuce nemá nová vydání, aktualizace dostáváte každých 30 minut pokud chcete. Nemusíte přeinstalovávat a migrovat na nová vydání, Váš počítač je stále aktuální
		\item Ze zdrojových kódů: můžete použít optimalizace přímo pro Váš stroj; 20\% rychlejší opengl vůči debianu na stejném PC atp. ; Nejvíce je to vidět na pomalejších strojích (zacate, atom, ...)
	\end{itemize}
\end{frame}
\begin{frame}{Výhody proti binárním distribucím 2}
	\begin{itemize}
		\item Možnost volby: nemáte rádi syslog-ng? Super nainstalujte si rsyslog. Nechcete v aplikacích podporu tisku? Jasně proste se vše překompiluje s vypnutým přepínačem cups...
		\item Téměř žádné distribuční patche: dostává se Vám vždy balíček tak jak si ho představoval vývojář při vydání.
		\item Život na hraně: podpora živých balíků a experimentálních vlastností v testovací větvi.
		\item Hardened projekt: zabezpečení Vašeho stroje o jakém se ostatním distribucím většinou nezdá ani v mokrých snech.
	\end{itemize}
\end{frame}

\begin{frame}{Nevýhody proti binárním distribucím}
	\begin{itemize}
		\item Složitá instalace: next next next finish... Opravdu ne :-)
		\item Rolling-updates: Počkat tohle bylo u výhod ne? Ano, ale neustálé změny verzí nemusí vyhovovat všem, když máte super vývojáře co si apikaci odladí pro php-5.2 tak nebudou rádi, když distribuční server aktualizuje na 5.4
		\item Pomalá instalace aplikací: pokud se nepoužijí binární balíčky (To jde? Jasně že jo!) tak kompilace ze zdrojových kódů chvíli trvá
		\item Testing strom je \uv{o hubu}: I když jej spousta lidí používá pro běžné účely je bezpečnější používat stabilní strom (i já ho používám)
	\end{itemize}
\end{frame}

\section{Je Gentoo pro mě?}
\begin{frame}{Kdy bych měl bych uvažovat o nasazení Gentoo?}
	\begin{itemize}
		\item Chci se pořádně naučit jak linux vlastně funguje
		\item Nevadí mi vyhledávat odpovědi na moje otázky na internetu
		\item Čtu si oznámení a jednám na základě jejich doporučení
	\end{itemize}
	Ve většině případů Gentoo lze velice dobře nasadit téměř do jakékoliv situace, pouze zvláštní rozhodnutí správce z toho časem udělají noční můru.
\end{frame}

\section{Jak se gentoo používá}
\begin{frame}{Pár tipů}
	\begin{itemize}
		\item Používat ladící údaje přidáním -g do C/CXXFLAGS a použitím splitdebug (viz man make.conf). Ladící informace pro celý můj bežící systém zabírají 3GB
		\item Kamarádit se s eix a používat jej na synchronizaci s eix-sync
		\item Aktualizovat alespoň jednou za měsíc strom portage a set @system \uv{emerge -NDu @system}
	\end{itemize}
\end{frame}

\section{Přispíváme}
\begin{frame}{Jak mohu přispívat?}
	\begin{itemize}
		\item Hlášením chyb na http://bugs.gentoo.org/
		\item Přispíváním kódů do repozitářů (stát se vývojářem, nebo přispívat jako proxy-maintainer)
		\item Financemi (donate čudl na http://www.gentoo.org/)
		\item Více informací zítra v podvečír na Gentoo miniconf na přednášce \uv{Getting involved more}
	\end{itemize}
\end{frame}

% -----------------------------------------------------------------------------
\section{Závěr}

\subsection{Další zdroje}
\begin{frame}{Další zdroje}
	\begin{itemize}
		\item Gentoo miniconf, která probíhá souběžně s touto konferencí o pár přednáškových sálů vedle
		\item http://www.gentoo.org/
		\item http://forums.gentoo.org/
		\item http://wiki.gentoo.org/
	\end{itemize}
\end{frame}

\subsection{Otázky}
\begin{frame}{Otázky a odpovědi}
	\begin{center}Otázky?\end{center}
\end{frame}

\subsection{Poděkování}
\begin{frame}{}
	\begin{center}Děkuji za pozornost\end{center}
\end{frame}
% -----------------------------------------------------------------------------
\end{document}
