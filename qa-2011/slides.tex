\documentclass{beamer}

\batchmode

\usepackage{amsmath,amssymb,enumerate,epsfig,bbm,calc,color,ifthen,capt-of}

\usetheme{Berlin}
\usecolortheme{mit}

%% <language_settings> %%
\usepackage[utf8x]{inputenc}
\usepackage[czech]{babel}
%% </language_settings> %%

%% Source code insertions
\usepackage{verbatim}

%% <change me> %%
\title{QA v Gentoo}
\author[Tomáš Chvátal]{Tomáš Chvátal $<$scarabeus@gentoo.org$>$}
\date{2011/04/19}
%% </change me> %%

%% <logo> %%
\usepackage[absolute,overlay]{textpos}
\setlength{\TPHorizModule}{1mm}
\setlength{\TPVertModule}{1mm}
\newcommand{\MyLogo}{
	\begin{textblock}{14}(110,7)
		\includegraphics[width=14mm,height=14mm]{gentoo-logo.png}
	\end{textblock}
}

\newcommand*\oldmacro{}
\let\oldmacro\insertshorttitle
\renewcommand*\insertshorttitle{
	\MyLogo
	\oldmacro\hfill
}
%% </logo> %%

%% <bgimage> %%
\usebackgroundtemplate{
	\includegraphics[width=\paperwidth,height=\paperheight]{gentoo-background-1.png}
}
%% </bgimage> %%

%% <ToC Before each section> %%
%\AtBeginSection[]{
%	\begin{frame}<beamer>
%		\frametitle{Přehled}
%		\tableofcontents[currentsection]
%	\end{frame}
%}
%% </ToC Before each section> %%

%% <each bullet on one slide> %%
%% better to use \pause on points where we really want to stop :)
%\beamerdefaultoverlayspecification{<+->}
%% </each bullet on one slide> %%
% -----------------------------------------------------------------------------
\begin{document}
% -----------------------------------------------------------------------------
\frame{\titlepage}
%\section[Přehled]{}
%\begin{frame}{Přehled}
%	\tableofcontents
%\end{frame}
% -----------------------------------------------------------------------------
\section{Představení}
\begin{frame}{About}
	\begin{center}Tomáš Chvátal\end{center}
	\begin{itemize}
		\item Vývojářem Gentoo od podzima 2008
		\item Členem Gentoo Council od ledna 2010
		\item Pracuje v týmech KDE, X11, QA, Sci-geosciences a Cluster
		\item V QA teamu pracuje na správném používání builscriptů a používání
tzv. ebuild-helpers
	\end{itemize}
\end{frame}
% -----------------------------------------------------------------------------
\section{Gentoo balíček}

\subsection{Ebuild}
\begin{frame}{Ebuild}
\verbatiminput{kwrite-4.6.2.ebuild}
\end{frame}
\begin{frame}{Global Local scope}
	\begin{center}Global\end{center}
	\begin{itemize}
		\item Globální proměnné
		\item Přístupné v metadatech
	\end{itemize}
	\begin{center}Local\end{center}
	\begin{itemize}
		\item Všechny proměnné by měli být pouze lokální
		\item spouštěna až při zavolání dané fáze
	\end{itemize}
\end{frame}
\begin{frame}{Ebuild fáze}
	\begin{itemize}
		\item pkg\_setup / pkg\_pretend
		\item src\_unpack / src\_prepare
		\item src\_configure / src\_compile
		\item src\_install
	\end{itemize}
\end{frame}

\subsection{Eclass}
\begin{frame}{Eclass}
Eclass je build script načítaný ebuildy pomocí příkazu inherit (podobný jako
příkaz source).
Vlastnosti těchto skriptů:
\begin{itemize}
	\item Neverzované (zatím)
	\item Změna eclass ovlivňuje sestavování všech balíků jí používajících
	\item Jsou bash skripty, které využívají vlastností dostupných k verzi 3.2
\end{itemize}
\end{frame}

\subsection{Ukázka}
\begin{frame}{Ukázka}
	\begin{center}Ukázka libav-9999.ebuild a git-2.eclass\end{center}
\end{frame}
% -----------------------------------------------------------------------------
\section{QA}

\subsection{Rozdělení QA v Gentoo}
\begin{frame}{Rozdělení QA v Gentoo}
	\begin{itemize}
	\item Interní QA pracující na:
		\begin{itemize}
			\item kvalitě buildscriptů (ebuild a eclass)
			\item kvalitě oprav (patchů), které přidáváme v Gentoo ({\it As close to upstream as possible})
		\end{itemize}
	\item Externí QA která se zaměřuje na:
		\begin{itemize}
			\item Správné používání buildscriptů (autotools, cmake,...)
			\item Nepoužívaní \uv{bundled} knihoven
			\item Kompilační testy a varování
			\item Testování reverzních závislostí
		\end{itemize}
	\end{itemize}
\end{frame}

% -----------------------------------------------------------------------------
\section{Závěr}
\subsection{Další zdroje}
\begin{frame}{Další zdroje}
	\begin{itemize}
		\item http://devmanual.gentoo.org/
		\item http://www.gentoo.org/proj/en/qa/
	\end{itemize}
\end{frame}
\subsection{Otázky}
\begin{frame}{Otázky a odpovědi}
	\begin{center}Otázky?\end{center}
	\begin{center}(Možná je i odpovím :P)\end{center}
\end{frame}

\subsection{Poděkování}
\begin{frame}{}
	\begin{center}Děkuji za pozornost\end{center}
\end{frame}
% -----------------------------------------------------------------------------
\end{document}
