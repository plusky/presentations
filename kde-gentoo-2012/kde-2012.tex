\documentclass{beamer}

\batchmode

\usepackage{amsmath,amssymb,enumerate,epsfig,bbm,calc,color,ifthen,capt-of}

\usetheme{Berlin}
\usecolortheme{mit}

%% <language_settings> %%
\usepackage[utf8x]{inputenc}
\usepackage[czech]{babel}
%% </language_settings> %%

%% Source code insertions
\usepackage{verbatim}

%% <change me> %%
\title{Gentoo KDE: stable, fresh, and bleeding edge!}
\author[Tomáš Chvátal]{Tomáš Chvátal $<$scarabeus@gentoo.org$>$}
\date{2012/10/20}
%% </change me> %%

%% <logo> %%
\usepackage[absolute,overlay]{textpos}
\setlength{\TPHorizModule}{1mm}
\setlength{\TPVertModule}{1mm}
\newcommand{\MyLogo}{
	\begin{textblock}{14}(110,7)
		\includegraphics[width=14mm,height=14mm]{gentoo-logo.png}
	\end{textblock}
}

\newcommand*\oldmacro{}
\let\oldmacro\insertshorttitle
\renewcommand*\insertshorttitle{
	\MyLogo
	\oldmacro\hfill
}
%% </logo> %%

%% <bgimage> %%
\usebackgroundtemplate{
	\includegraphics[width=\paperwidth,height=\paperheight]{gentoo-background-1.png}
}
%% </bgimage> %%

%% <ToC Before each section> %%
%\AtBeginSection[]{
%	\begin{frame}<beamer>
%		\frametitle{Přehled}
%		\tableofcontents[currentsection]
%	\end{frame}
%}
%% </ToC Before each section> %%

%% <each bullet on one slide> %%
%% better to use \pause on points where we really want to stop :)
%\beamerdefaultoverlayspecification{<+->}
%% </each bullet on one slide> %%
% -----------------------------------------------------------------------------
\begin{document}
% -----------------------------------------------------------------------------
\frame{\titlepage}
%\section[Přehled]{}
%\begin{frame}{Přehled}
%	\tableofcontents
%\end{frame}
% -----------------------------------------------------------------------------
\section{Introduction}

\begin{frame}{Who the hell is Tomáš Chvátal}
	\begin{itemize}
		\item Council member (currently second term)
		\item KDE team member (former lead)
		\item Libreoffice maintainer
	\end{itemize}
\end{frame}

% -----------------------------------------------------------------------------
\section{What is KDE?}

\begin{frame}
	\begin{itemize}
		\item Fancy desktop environment with many features.
		\item Bloated thing wrt database usage.
	\end{itemize}
\end{frame}

\subsection{So how to make it less pain and actually usable}
\begin{frame}
	\begin{itemize}
		\item Make the nepomuk (aka semantic-desktop) optional for those who preffer to use their system rather than cache its content all the time.
		\item Make akonadi also optional and allow installation of pre-akonadi versions to have something really usable :-)
		\item Make all this fun run on hardened and so make it bit safer.
	\end{itemize}
\end{frame}

\section{Running KDE on gentoo}
\begin{frame}
	\begin{itemize}
		\item Use desktop or desktop/kde profile
		\item Enable devicekit and consolekit useflag
		\item Select the packages that are actually needed
	\end{itemize}
\end{frame}

\subsection{What packages would I need}
\begin{frame}
	\begin{itemize}
		\item kde-base/kde-meta: Not really, this is metapackage that includes all of them.
		\item kde-base/kdebase-meta and kde-base/kdebase-runtime-meta: Basic subset on which one can install additional packates.
		\item kde-base/*-meta: These are jont packages per various category.
	\end{itemize}
\end{frame}

\subsection{Interesting useflags}
\begin{frame}
	\begin{itemize}
		\item semantic-desktop: useflag removing all akonadi, virtuoso and nepomuk dependencies.
		\item udisks/upower: enable or disable dependency over these tools, it is runtime detected tho.
	\end{itemize}
\end{frame}

\subsection{Bleedingness}
\begin{frame}
	\begin{center}Stable\end{center}
	\begin{itemize}
		\item Bit older than latest released stable version from upstream, but tested a lot to ensure its stability.
		\item Usually new stable version is done every other month.
	\end{itemize}
\end{frame}

\begin{frame}
	\begin{center}Testing\end{center}
	\begin{itemize}
		\item Always 0 day version bump with latest upstream released stuff.
		\item Usually contains a lots of issues that upstream didnt care enough to fix.
		\item This stuff is taken from overlay by fancy update script in bash i wrote 4 years ago.
	\end{itemize}
\end{frame}

\begin{frame}
	Overlay - aka snapshots / live stuff ; let me break your boxen TM
	\begin{itemize}
		\item Usually live versions of master branch and currently produced branch (eg 4.8 when 4.9 is in development).
		\item The overlay is the place where all the development is done.
		\item The branch ebuilds are usually just the bugfixes to next release.
		\item Live ebuilds is where magic happens, and you will usually cry for help when stuff breaks.
	\end{itemize}
\end{frame}

\section{What can I do for KDE in Gentoo}
\begin{frame}
	\begin{itemize}
		\item Join #gentoo-kde on freenode.net and chat with us about your secret plans.
		\item Clone up the git repository and help ohers by using live ebuilds and fix the issues.
		\item Fire up bugzilla and help us closing issues (we like to have just few bugs, something around 100 is perfect).
	\end{itemize}
\end{frame}

\section{KDE SC 5}
\begin{frame}
	\begin{center}Bit of discussion about KDE future.\end{center}
\end{frame}

% -----------------------------------------------------------------------------
\section{Wrap-up}

\subsection{More resources}
\begin{frame}{Additional links to learn from}
	\begin{itemize}
		\item http://wiki.gentoo.org/wiki/KDE
		\item http://www.gentoo.org/proj/en/desktop/kde/
	\end{itemize}
\end{frame}

\subsection{Questions}
\begin{frame}{Q\&A session}
	\begin{center}Nobody wants to ask anything. Right?\end{center}
\end{frame}

\subsection{Thanks}
\begin{frame}{Thank you}
	\begin{center}Thank you for your attention!\end{center}
\end{frame}
% -----------------------------------------------------------------------------
\end{document}
